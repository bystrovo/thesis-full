\clearpage

\paragraph{Developing a high performing interface}

There is a multitude of studies that analyze user interface design and user emotion. In following I will summarize them in respect to UI features tailored towards high performance

\paragraph{Literature review}

A paper by L. Arockiam et al \cite{Arockiam2013} describes that, based on previous research and their own findings optimal ui can be achieved based on a person's personality traits. We will limit ourselves to a universal set of changes that should evoke high arousal, positive valence and therefore have an effect on learning outcome, rather than look for their preference. Nevertheless it is worth mentioning...

Color - PHYSIOLOGICAL

Wilson (1966) reported higher GSR measurements for red as opposed to green, and Nourse and Welsh (1971) reported higher GSR readings for violet than green. Using 24 male college students as subjects and saturated samples of red, yellow, green, and blue as stimulus materials, Jacobs and Hustmyer (1974) found that red was significantly more arousing than either yellow or blue, and green more than blue. Using 40 undergraduate students as subjects, Jacobs and Suess (1975) found that red and yellow resulted in higher anxiety state scores than blue or green when measured by the StateTrait Anxiety Inventory. Bloomer (1976) also reported that red increases heart rate. There appears to be some evidence that spectral extremes, especially red, cause greater arousal than mid-spectral colors. This may relate to the fact that wavelengths at the extremes of the spectrum, such as red and violet, focus at different points in the eye than wavelengths at the middle of the spectrum. \cite{Pert1996}

Color - PSYCHOLOGICAL

The selected order of preference was: (1) blue, (2) red, (3) green, (4) violet, (5) orange, (6) yellow. This selection agreed with the average rankings of color preference among 21,060 subjects reported in earlier investigations. The order was the same for all races and for men and women with one exception. Men chose orange over yellow whereas women chose yellow over orange. \cite{Pert1996}

no significant differ- ences between men and women or between subjects of different races.\cite{Pert1996}

girls of all ages preferred higher value, brighter, colors as compared to boys (Child, et al., 1968) \cite{Pert1996} (study among school age children from 1 to 12 grade)

Results on the evaluation scale supported the general preference for cool colors(bl~e and green) as compared with warm colors(red and yellow) and agree with previous findings by Adams and Osgood (1973) that red is a potent color while gray and black have low potency. \cite{Pert1996}

In a color-effectiveness study conducted at Fort Monmouth, typical army training procedures were used with 11 different television lessons (Kanner and Rosenstein, 1960). No significant
differences were found in learning between
monochrome and colored versions \cite{Pert1996}

Schaie (1966) pointed out that color prefer-
ences vary from individual to individual and relate to personality. \cite{Pert1996}

(so far black and white and color has not yielded any significant difference in learning) but "colored mate- rials are preferred by learners." \cite{Pert1996}

1972: those who viewed colored transparencies had had a more positive attitude toward transparencies than \cite{Pert1996}

In a research study on color coding, Lamberski and Dwyer (1983) concluded that color is an attention-getting device that can provide measurable effects on learning that cannot be accounted for by words and labels. \cite{Pert1996}

Search Tasks: 
gain in efficiency, indicated by decreased search time, with codes of up to five colors \cite{Pert1996}

Color was found to be useful in grouping information

color versions resulted in higher recognition- memory scores (immediate recognition memory test)\cite{Pert1996}

as the variable of visual complexity increases, so does the degree of recall. (Berry (1991a)) \cite{Pert1996}

The key factor relating to color and cogni-
tive learning seems to be that it is of value when it emphasizes relevant cues, is used as a coding device, or when it is a part of the con- tent to be learned (Dwyer and Lamberski, 1982- 83; Levie, 1973; Pruisner, 1993; Wedell \& Alden, 1973). \cite{Pert1996}

Non-objective Measures

(Scanlon, 1970). Scanlon suggests that the color versions (Grey Cup football game) create emotional effects that detract from attention to details.







Colors at the ends of the spectrum, red and
violet, seem to result in greater arousal, and
colors in the middle of the spectrum, yellow,
green, cyan, seem to be best for discriminating
detail. \cite{Pert1996}