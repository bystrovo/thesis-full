\paragraph{Experiment type} A quantitative experiment is proposed to obtain results for this study. Field tests, combined with limited laboratory tests are conducted. E-learning is, essentially, a field setting and can

\paragraph{Participant sourcing} 
This study includes participants attained through multiple sources:
\begin{itemize}
	\item{Local university:} On-site supervised experiments were conducted on a limited scale to facilitate a clean sample of participants and uncover problems during the study
	
	\item{Local workplace:} On-site semi-supervised experiments are conducted on a limited scale in Berlin area in Germany to facilitate a more diverse sample while keeping controlled environment conditions, similar to university experiments.
	
	\item{Social media:} Remote participants are invited to participate in unsupervised experiments on their own. Channels such as social media, interest groups and university mailing lists are used.
	
	\item{Mechanical turk:} MTurk is one of popuar web services to source participants. Previous research has shown that it can be considered a reliable platform for conducting objective studies. MTurk participants receive monetary compensation for participation \cite{Buhrmester2011a}. Mturk participants are excluded from motivation with a chance on winning a voucher prize (they do not see the field and information box about it), instead regular participant reward through MTurk is used.
	
\end{itemize}

Local and social media experiments provided additional motivation of participation with the promise of a chance to win a 20 Euro Amazon-voucher.